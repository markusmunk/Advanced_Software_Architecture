\documentclass[conference]{IEEEtran}
\IEEEoverridecommandlockouts
% The preceding line is only needed to identify funding in the first footnote. If that is unneeded, please comment it out.
\usepackage{cite}
\usepackage{amsmath,amssymb,amsfonts}
\usepackage{algorithmic}
\usepackage{graphicx}
\usepackage{textcomp}
\usepackage{xcolor}

\usepackage{multirow}
\usepackage{rotating}

\usepackage{mdframed}
\usepackage{hyperref}
\usepackage{tikz}
\usepackage{makecell}
\usepackage{tcolorbox}
\usepackage{amsthm}
%\usepackage[english]{babel}
\usepackage{pifont} % checkmarks
%\theoremstyle{definition}
%\newtheorem{definition}{Definition}[section]


\usepackage{listings}
\lstset
{ 
    basicstyle=\footnotesize,
    numbers=left,
    stepnumber=1,
    xleftmargin=5.0ex,
}


%SCJ
\usepackage{subcaption}
\usepackage{array, multirow}
\usepackage{enumitem}


\def\BibTeX{{\rm B\kern-.05em{\sc i\kern-.025em b}\kern-.08em
    T\kern-.1667em\lower.7ex\hbox{E}\kern-.125emX}}
\begin{document}

%\IEEEpubid{978-1-6654-8356-8/22/\$31.00 ©2022 IEEE}
% @Sune:
% Found this suggestion: https://site.ieee.org/compel2018/ieee-copyright-notice/
% I have added it - you can see if it fulfills the requirements

%\IEEEoverridecommandlockouts
%\IEEEpubid{\makebox[\columnwidth]{978-1-6654-8356-8/22/\$31.00 ©2022 IEEE %\hfill} \hspace{\columnsep}\makebox[\columnwidth]{ }}
                                 %978-1-6654-8356-8/22/$31.00 ©2022 IEEE
% copyright notice added:
%\makeatletter
%\setlength{\footskip}{20pt} 
%\def\ps@IEEEtitlepagestyle{%
%  \def\@oddfoot{\mycopyrightnotice}%
%  \def\@evenfoot{}%
%}
%\def\mycopyrightnotice{%
%  {\footnotesize 978-1-6654-8356-8/22/\$31.00 ©2022 IEEE\hfill}% <--- Change here
%  \gdef\mycopyrightnotice{}% just in case
%}

      
\title{Reflection Report Template\\
}

\author{
    \IEEEauthorblockN{
        Student 1
    }
    \IEEEauthorblockA{
        University of Southern Denmark, SDU Software Engineering, Odense, Denmark \\
        Email: student1@mmmi.sdu.dk
    }
}


%%%%

%\author{\IEEEauthorblockN{1\textsuperscript{st} Blinded for review}
%\IEEEauthorblockA{\textit{Blinded for review} \\
%\textit{Blinded for review}\\
%Blinded for review \\
%Blinded for review}
%\and
%\IEEEauthorblockN{2\textsuperscript{nd} Blinded for review}
%\IEEEauthorblockA{\textit{Blinded for review} \\
%\textit{Blinded for review}\\
%Blinded for review \\
%Blinded for review}
%\and
%\IEEEauthorblockN{3\textsuperscript{nd} Blinded for review}
%\IEEEauthorblockA{\textit{Blinded for review} \\
%\textit{Blinded for review}\\
%Blinded for review \\
%Blinded for review}
%}

%%%%
%\IEEEauthorblockN{2\textsuperscript{nd} Given Name Surname}
%\IEEEauthorblockA{\textit{dept. name of organization (of Aff.)} \\
%\textit{name of organization (of Aff.)}\\
%City, Country \\
%email address or ORCID}


\maketitle
\IEEEpubidadjcol


\section{Contribution}

\section{Burger 4.0 Production System}

\subsection{Use Cases}

\begin{itemize}
    \item \textbf{Order Placement:} Customers can place orders through the GUI, selecting from the menu options.
    
    \item \textbf{Order Tracking:} Customers can view the progress of their orders in real-time through the GUI.
    
    \item \textbf{Ingredient Preparation:} Stations prepare and process the ingredients required for burger, fries, and soda orders.
    
    \item \textbf{Burger Assembly:} A dedicated station assembles the burger orders according to the customer's specifications.
    
    \item \textbf{Robotic Transport:} Robots autonomously move ingredients and assembled items between stations.
    
    \item \textbf{Packaging:} A packaging station ensures that orders are appropriately packaged before delivery.
    
    \item \textbf{Order Delivery:} Robots, similar to MIR-bots, deliver completed orders to the pick-up station for customers.
\end{itemize}

\subsection{Required Stations}

\begin{enumerate}
    \item \textbf{Ordering GUI Station:} This station provides the user interface for customers to place orders and track their progress.
    
    \item \textbf{Ingredient Preparation Stations:} Multiple stations equipped with various technologies for preparing burger ingredients, fries, and soda.
    
    \item \textbf{Burger Assembly Station:} A specialized station for assembling burgers with precision.
    
    \item \textbf{Robotic Transport System:} Autonomous robots equipped with sensors and navigation systems for moving ingredients and assembled items between stations.
    
    \item \textbf{Packaging Station:} A station responsible for packaging orders securely.
    
    \item \textbf{Order Pick-up Station:} Customers collect their orders from this station, where robot deliveries are made.
\end{enumerate}

\subsection{Relevant Technologies}

\begin{itemize}
    \item \textbf{IoT Sensors:} Utilized in ingredient preparation stations for monitoring and controlling cooking processes.
    
    \item \textbf{Robotic Process Automation (RPA):} Employed in the burger assembly station for precise and efficient assembly of burger components.
    
    \item \textbf{Machine Vision Systems:} Used in robotic transport systems for object recognition and navigation.
    
    \item \textbf{Artificial Intelligence (AI):} AI algorithms can optimize the routing and coordination of robots within the production system.
    
    \item \textbf{RFID Technology:} Implemented for tracking and managing inventory of ingredients and packaging materials.
    
    \item \textbf{IoT Communication Protocols:} Utilized for real-time data exchange between stations and the central control system.
    
    \item \textbf{Human-Machine Interface (HMI):} Integrated into the GUI for user-friendly order placement and tracking.
    
    \item \textbf{Event-Driven Architecture:} Enables communication between stations and updates the GUI with order progress information.
\end{itemize}

\subsection{Assumptions}

\begin{itemize}
    \item The system operates in a controlled environment, simulating the production process without physical machines.
    
    \item Sufficient computing resources and network connectivity are available to support the system's operations.
    
    \item Real-time data exchange and communication among stations are reliable and low-latency.
\end{itemize}

\subsection{Challenges and Considerations}

\begin{itemize}
    \item Scalability: Consider the system's scalability to accommodate varying order volumes and production demands.
    
    \item Fault Tolerance: Implement redundancy and failover mechanisms to ensure uninterrupted production.
    
    \item Data Security: Safeguard customer data and system integrity against potential cyber threats.
    
    \item Maintenance and Upkeep: Plan for regular maintenance and updates of the simulated system.
    
    \item Integration: Ensure seamless integration of technologies and stations for efficient production.
\end{itemize}

\section{Discussion}



\section{Reflection}

Reflection


\section{Conclusion}


\bibliographystyle{IEEEtran}
\bibliography{references}
\vspace{12pt}
\end{document}
